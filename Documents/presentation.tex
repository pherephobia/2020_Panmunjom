\documentclass[letterpaper,10pt,twocolumn,twoside,]{pinp}

%% Some pieces required from the pandoc template
\providecommand{\tightlist}{%
  \setlength{\itemsep}{0pt}\setlength{\parskip}{0pt}}

% Use the lineno option to display guide line numbers if required.
% Note that the use of elements such as single-column equations
% may affect the guide line number alignment.

\usepackage[T1]{fontenc}
\usepackage[utf8]{inputenc}

% pinp change: the geometry package layout settings need to be set here, not in pinp.cls
\geometry{layoutsize={0.95588\paperwidth,0.98864\paperheight},%
  layouthoffset=0.02206\paperwidth, layoutvoffset=0.00568\paperheight}

\definecolor{pinpblue}{HTML}{185FAF}  % imagecolorpicker on blue for new R logo
\definecolor{pnasbluetext}{RGB}{101,0,0} %


\usepackage{ntheorem}
\theoremseparator{:}
\newtheorem{hyp}{Hypothesis}

\title{Persistence and Change in Attitudes Toward Unification with North
Korea}

\author[a]{Jaeyoung Hur}

  \affil[a]{Global leaders College, Yonsei University, Seoul, ROK}

\setcounter{secnumdepth}{0}

% Please give the surname of the lead author for the running footer
\leadauthor{Jaeyoung Hur}

% Keywords are not mandatory, but authors are strongly encouraged to provide them. If provided, please include two to five keywords, separated by the pipe symbol, e.g:
 

\begin{abstract}

\end{abstract}

\dates{This version was compiled on \today} 


% initially we use doi so keep for backwards compatibility
% new name is doi_footer

\pinpfootercontents{Presentation Script}

\begin{document}

% Optional adjustment to line up main text (after abstract) of first page with line numbers, when using both lineno and twocolumn options.
% You should only change this length when you've finalised the article contents.
\verticaladjustment{-2pt}

\maketitle
\thispagestyle{firststyle}
\ifthenelse{\boolean{shortarticle}}{\ifthenelse{\boolean{singlecolumn}}{\abscontentformatted}{\abscontent}}{}

% If your first paragraph (i.e. with the \dropcap) contains a list environment (quote, quotation, theorem, definition, enumerate, itemize...), the line after the list may have some extra indentation. If this is the case, add \parshape=0 to the end of the list environment.


\hypertarget{motivations-and-research-questions}{%
\subsection{Motivations and Research
Questions}\label{motivations-and-research-questions}}

The inter-Korean relationship faced several historical events in 2018.
After the new administration pursues cooperative policies with North
Korea, North Korea sent athletes to the Pyeongchang Winter Olympics. It
seems a signal that South Korea and North Korea enter into the new state
of their relationship. Also, the two Korea's leaders meet at Panmunjom
to declare peace, prosperity, and unification.

The dramatic changes in the Inter-Korean relationship made South Koreans
expect a `better future.' however, we need to remind that the
inter-Korean relationship has varied by different administrations.
Although Moon Jae-in administration turned to the N.K. cooperative
stance, whether it will succeed, is not guaranteed.

Existing studies explain the impacts of `historical events' between
North and South Koreas. However, Koreans have experiences inconsistent
policies and N.K.'s Intermittent provocation. Here, we ask whether the
4.27 Panmunjom declaration leads to fundamental changes for the
inter-Korean relationship. In particular, we examine whether South
Koreans' attitudes and perceptions toward North Korea and unification
have changed.

\hypertarget{literature-reviews}{%
\subsection{Literature Reviews}\label{literature-reviews}}

Previous studies present various factors that are expected to affect the
attitudes toward North Korea and Unification. For a long time, the
nationalism-based legitimacy of unification explained why we ``should''
pursue the unification. However, as the time of division lasts,
political factors or cost-benefits approaches emerge to explain South
Koreans' attitudes. Some scholars also argue that a group of people who
have similar experiences will show similar attitudes, and those
attitudes might vary across the groups (generations). Although previous
literature has provided various explanations, they do not consider the
possible impact outside, such as external events. In the inter-Korean
relationship, we have observed several ``unexpected'' political events
that drastically shift the inter-Korean relationship.

Some people expect the Panmunjom Declaration will change the
inter-Korean relationship. In fact, various media suggested that the
Korean peninsula prospect would be better, and after the declaration,
President Moon's approval exceeded 80\%.

However, does the declaration bring fundamental changes to us as we
expected? South Koreans have experienced varying policy cycles by
different administrations. Conservative administrations tended to take a
strong stance against North Korea, while liberal administrations led the
engagement policy as a basis for reconciliation and cooperation. It
means that members of South Korea have already experienced a similar
type of event and its failure.

\hypertarget{research-design}{%
\subsection{Research Design}\label{research-design}}

We examine the changes in attitudes toward North Korea. We expect that
generations mainly experienced tensions and conflicts between South and
North for a long time are more likely to show more negative attitudes
toward North Korea and unification. Otherwise, generations under
cooperative moods between South and North can expect an improvement of
inter-Korean relationships. Thus, we draw the first hypothesis as
follows:

\begin{hyp}\label{hyp1} The younger generation show more negative attitudes toward North Korea than other generations, even after the Panmunjom Declaration. \end{hyp}

Next, we also examine the changes in attitudes toward unification.
According to recent studies, the cost and benefits of unification are
dominant explanations to justify unification. However, the utilities of
unification can vary depending on the conditions.

When we state the Panmunjom Declaration is a positive signal about
unification prospects, the respondents will show different attitudes
conditional on the varying prospects. The second hypothesis of this
project as follows:

\begin{hyp}\label{hyp2} The prospects for unification affect the need for unification differently across generations. \end{hyp}

In particular, we expect that the more positive prospects one expects,
the more likely, the older generations will favor unification. However,
we also expect that even the more positive prospects they expect, the
younger generations are less likely to support unification.

\hypertarget{data}{%
\subsubsection{Data}\label{data}}

Our sample is from the KBS National unification consciousness survey of
2018. It is surveyed from August 3rd to 7th in 2018. The size of the
sample is 1,000 over the age of 19. We think the sample is the most
relevant to examine the research question due to the least time
difference from the Panmunjom Declaration.

\hypertarget{dependent-variable}{%
\subsubsection{Dependent variable}\label{dependent-variable}}

\hypertarget{attitudes-toward-north-korea-leadership-and-general-state}{%
\paragraph{Attitudes toward North Korea: Leadership and General
State}\label{attitudes-toward-north-korea-leadership-and-general-state}}

The first dependent variable is the attitudes toward North Korea. We
have two indicators to show it. The first is the attitudes toward North
Korean leadership, and the second is the attitudes toward general North
Korea as a state.

When we looked at the distribution of attitudes toward North Korea after
the Panmunjom Declaration, we can see a difference between the attitudes
toward North Korean leadership and North Korea in general. The data
shows that after the Panmunjom Declaration, respondents are still
dissatisfied with the North Korean leadership. At the same time,
however, respondents perceive North Korea as an inevitable entity to
interact to solve the inter-Korean relationship and other problems
concerning the Korean peninsula.

\hypertarget{attitudes-toward-unification}{%
\paragraph{Attitudes toward
Unification}\label{attitudes-toward-unification}}

The second dependent variable is the attitudes toward unification. The
most frequent response is that respondents prefer unification if it
creates no burdens (45.5\%). Only 8.9\% of respondents say that they do
not prefer unification. A quarter of the respondents answer that they
prefer the status quo (25.2\%), and only 20.4\% say they prefer
unification unconditionally.

\hypertarget{key-explanatory-variables-and-controls}{%
\subsubsection{Key explanatory variables and
controls}\label{key-explanatory-variables-and-controls}}

One of the main explanatory variables in this study is generational. We
investigate how the generations are associated with the attitudes toward
North Korea and unification after the Panmunjom Declaration. Generations
are categorical variables measured in 10-year units based on age.

The other explanatory variable of interest is the prospect of
unification. The longer the period of unification, the more negative the
prospect. We coded that the most negative prospect as 1, and the most
positive prospect as 6.

Finally, we control for other variables that can potentially affect the
attitudes toward North Korea and unification. The variables include the
regions where respondents live, gender, education level, income level,
evaluations of North Korea policies under the Jae-In Moon
administration, all drawn from the KBS survey data.

\hypertarget{methods}{%
\subsubsection{Methods}\label{methods}}

As the dependent variable is ordered and discrete, we use the ordered
logistic regression model. However, as the coefficients of ordered
logistic regression is not intuitive to understand the results, we will
present the first differences in predicted probabilities and predicted
probabilities of key explanatory variables' effects.

\hypertarget{empirical-findings}{%
\subsection{Empirical Findings}\label{empirical-findings}}

\hypertarget{model-for-the-hypothesis}{%
\subsubsection{\texorpdfstring{Model for the hypothesis
\ref{hyp1}}{Model for the hypothesis }}\label{model-for-the-hypothesis}}

The first model is about attitudes toward North Korea after the
Panmunjom Declaration. As the general N.K. attitudes show no significant
differences across the generations. Here, we present the model of
attitudes toward North Korean leadership.

The 40s and the 50s show statistically significant differences compared
to the 20s. It means that the 40s and the 50s are more likely to favor
North Korean leadership than the 20s.

The plot of the first differences in predicted probabilities shows that
the 40s and the 50s are more likely to be satisfied with the North
Korean leadership while they are less likely to be dissatisfied with the
North Korean leadership.

\hypertarget{model-for-the-hypothesis-1}{%
\subsubsection{\texorpdfstring{Model for the hypothesis
\ref{hyp2}}{Model for the hypothesis }}\label{model-for-the-hypothesis-1}}

Next, the second model shows how the attitudes toward unification by
different generations are conditional on unification prospects. The
results show that the 50s and 60s are more likely to be favorable toward
unification than the 20s. The 30s and 40s are not significantly
different from the 20s.

We estimate the predicted probabilities of each generation group that
how their choices of the unification are conditional on unification
prospects. The plot of the 50s and 60s shows that as the prospects
become positive, they are less likely to choose ``not preferred'' and
``Status quo.'' Otherwise, as the prospects become positive, they are
more likely to support unification.

Then, how about the 20s, 30s, and 60+? Interestingly, the trends are not
so different in the predicted probabilities. The only difference can be
seen at the panel of ``preferred without burdens.'' The 20s, 30s, and
40s show more significant variations in the predicted probabilities as
unification prospects vary.

\hypertarget{conclusion-and-implications}{%
\subsection{Conclusion and
Implications}\label{conclusion-and-implications}}

The hypothesis \ref{hyp1} is partially confirmed. The younger
generations show more negative attitudes only toward North Korean
leadership, as well as 60+ does. However, the 40s and 50s are more
likely to be favorable toward North Korean leadership than the 20s.

The hypothesis \ref{hyp2} is confirmed. The 30s and 40s are
statistically insignificant compared to the 20s. However, the 50s and
60+ are more likely to be favorable toward unification as unification
prospects become positive compared to the 20s.

It is difficult to say that the event improves unification prospects
(Figure of predicted probabilities of unification model). In terms of
unification, even when unification prospects increase, people in their
20s are more likely to have the most negative attitudes toward
unification on average. The younger generations and the 40s seem to
estimate the costs and benefits by the extent to the expected
unification timing.

This study also shows that the significant events driving changes in
inter-Korean relations can be made on the supply side, but we should
also pay attention to the changes that appear in the ordinary people,
who are the consumers, and other parties of the inter-Korean
relationship. As a supplier in the unification and North Korea issues,
the government provides one-sided policy options and should also embrace
the people's perceptions and attitudes, which are sensitive to changes
in the real world. Simultaneously, to establish the justification logic
of unification based on generational changes, it is necessary to ensure
long-term consistency in at least unification and North Korea policy.

%\showmatmethods


\bibliography{parkhur2020}
\bibliographystyle{jss}



\end{document}
